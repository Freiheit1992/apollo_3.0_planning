\documentclass[a4paper,11pt]{article}  %声明文档类型为article
\usepackage{amsmath}
\begin{document}   % 声明文档开始
- QP-Spline-ST-Speed Optimizer

**Tip**: to read the equations in the document, you are recommended to use Chrome with [a plugin](https://chrome.google.com/webstore/detail/tex-all-the-things/cbimabofgmfdkicghcadidpemeenbffn) or copy the latex equation to [an online editor](http://www.hostmath.com/)

-- 1  Definition 

After finding a path in QP-Spline-Path, Apollo converts all obstacles on the path and the ADV (autonomous driving vehicle) into an path-time (S-T) graph, which represents that the station changes over time along the path. The speed optimization task is to find a path on the S-T graph that is collision-free and comfortable. 

Apollo uses splines to represent speed profiles, which are lists of S-T points in S-T graph. Apollo leverages Quadratic programming to find the best profile. The standard form of QP problem is defined as: 

$$
minimize \frac{1}{2} \cdot x^T \cdot H \cdot x + f^T \cdot x 
\\
s.t. LB \leq x \leq UB
\\
A_{eq}x = b_{eq}
\\
Ax \leq b
$$


-- 2  Objective function

--- 2.1  Get spline segments

Split the S-T profile into **n** segments. Each segment trajectory is defined by a polynomial.

--- 2.2  Define function for each spline segment

Each segment ***i*** has an accumulated distance $d_i$ along a reference line. And the trajectory for the segment is defined as a polynomial of degree five by default. The degree of the polynomials are adjustable by configuration parameters.


$$
s = f_i(t) 
  = a_{0i} + a_{1i} \cdot t + a_{2i} \cdot t^2 + a_{3i} \cdot t^3 + a_{4i} \cdot t^4 + a_{5i} \cdot t^5
$$


--- 2.3  Define  objective function of optimization for each segment

Apollo first defines $cost_1$ to make the trajectory smooth: 

$$
cost_1 = \sum_{i=1}^{n} \Big( w_1 \cdot \int\limits_{0}^{d_i} (f_i')^2(s) ds + w_2 \cdot \int\limits_{0}^{d_i} (f_i'')^2(s) ds + w_3 \cdot \int\limits_{0}^{d_i} (f_i^{\prime\prime\prime})^2(s) ds \Big)
$$


Then Apollo defines $cost_2$ as the difference between the final S-T trajectory and the cruise S-T trajectory (with given speed limits — m points):

$$
cost_2 = \sum_{i=1}^{n}\sum_{j=1}^{m}\Big(f_i(t_j)- s_j\Big)^2
$$

Similarly, Apollo defines $cost_3$ that is the difference between the first S-T path and the follow S-T path (o points):

$$
cost_3 = \sum_{i=1}^{n}\sum_{j=1}^{o}\Big(f_i(t_j)- s_j\Big)^2
$$

Finally, the objective function is defined as: 

$$
cost = cost_1 + cost_2 + cost_3
$$


-- 3  Constraints  

--- 3.1 The init point constraints

Given the assumption that the the first point is ($t0$, $s0$), and $s0$ is on the planned path $f_i(t)$, $f'i(t)$, and $f_i(t)''$ (position, velocity, acceleration).  Apollo converts those constraint into QP equality constraints:

$$
A_{eq}x = b_{eq}
$$


--- 3.2  Monotone constraint

The path must be monotone, e.g., the vehicle can only drive forward. 

Sample **m** points on the path, for each $j$ and $j-1$ point pairs ($j\in[1,...,m]$): 

If the two points on the same spline $k$:

$$
\begin{vmatrix}  1 & t_j & t_j^2 & t_j^3 & t_j^4&t_j^5 \\ \end{vmatrix} 
\cdot 
\begin{vmatrix}  a_k \\ b_k \\ c_k \\ d_k \\ e_k \\ f_k  \end{vmatrix} 
> 
\begin{vmatrix}  1 & t_{j-1} & t_{j-1}^2 & t_{j-1}^3 & t_{j-1}^4&t_{j-1}^5 \\ \end{vmatrix}  
\cdot 
\begin{vmatrix}  a_{k} \\ b_{k} \\ c_{k} \\ d_{k} \\ e_{k} \\ f_{k}  \end{vmatrix}
$$

 If the two points on the different spline $k$ and $l$:

$$
\begin{vmatrix}  1 & t_j & t_j^2 & t_j^3 & t_j^4&t_j^5 \\ \end{vmatrix} 
\cdot 
\begin{vmatrix}  a_k \\ b_k \\ c_k \\ d_k \\ e_k \\ f_k  \end{vmatrix} 
> 
\begin{vmatrix}  1 & t_{j-1} & t_{j-1}^2 & t_{j-1}^3 & t_{j-1}^4&t_{j-1}^5 \\ \end{vmatrix}  
\cdot 
\begin{vmatrix}  a_{l} \\ b_{l} \\ c_{l} \\ d_{l} \\ e_{l} \\ f_{l}  \end{vmatrix}
$$




--- 3.3  Joint smoothness  constraints

This constraint is designed to smooth the spline joint.  Given the assumption that two segments, $seg_k$ and $seg_{k+1}$, are connected, and the accumulated **s** of segment $seg_k$ is $s_k$,  Apollo calculates the constraint equation as: 

$$
f_k(t_k) = f_{k+1} (t_0)
$$

Namely:

$$
\begin{vmatrix} 
 1 & t_k & t_k^2 & t_k^3 & t_k^4&t_k^5 \\
 \end{vmatrix} 
 \cdot 
 \begin{vmatrix} 
 a_{k0} \\ a_{k1} \\ a_{k2} \\ a_{k3} \\ a_{k4} \\ a_{k5} 
 \end{vmatrix} 
 = 
\begin{vmatrix} 
 1 & t_{0} & t_{0}^2 & t_{0}^3 & t_{0}^4&t_{0}^5 \\
 \end{vmatrix} 
 \cdot 
 \begin{vmatrix} 
 a_{k+1,0} \\ a_{k+1,1} \\ a_{k+1,2} \\ a_{k+1,3} \\ a_{k+1,4} \\ a_{k+1,5} 
 \end{vmatrix}
$$

Then

$$
\addtocounter{MaxMatrixCols}{10}
\begin{vmatrix} 
 1 & t_k & t_k^2 & t_k^3 & t_k^4&t_k^5 &  -1 & -t_{0} & -t_{0}^2 & -t_{0}^3 & -t_{0}^4&-t_{0}^5\\
 \end{vmatrix} 
 \cdot 
 \begin{vmatrix} 
 a_{k0} \\ a_{k1} \\ a_{k2} \\ a_{k3} \\ a_{k4} \\ a_{k5} \\ a_{k+1,0} \\ a_{k+1,1} \\ a_{k+1,2} \\ a_{k+1,3} \\ a_{k+1,4} \\ a_{k+1,5}   
 \end{vmatrix} 
 = 0
$$

The result is $t_0$ = 0 in the equation.

Similarly calculate the equality constraints for 

$$
f'_k(t_k) = f'_{k+1} (t_0)
\\
f''_k(t_k) = f''_{k+1} (t_0)
\\
f'''_k(t_k) = f'''_{k+1} (t_0)
$$


--- 3.4  Sampled points for boundary constraint

Evenly sample **m** points along the path, and check the obstacle boundary at those points.  Convert the constraint into QP inequality constraints, using:

$$
Ax \leq b
$$

Apollo first finds the lower boundary $l_{lb,j}$ at those points ($s_j$, $l_j$) and  $j\in[0, m]$ based on the road width and surrounding obstacles. Then it calculates the inequality constraints as:

$$
\begin{vmatrix} 
 1 & t_0 & t_0^2 & t_0^3 & t_0^4&t_0^5 \\
  1 & t_1 & t_1^2 & t_1^3 & t_1^4&t_1^5 \\
 ...&...&...&...&...&... \\
 1 & t_m & t_m^2 & t_m^3 & t_m^4&t_m^5 \\
 \end{vmatrix} \cdot \begin{vmatrix} a_i \\ b_i \\ c_i \\ d_i \\ e_i \\ f_i \end{vmatrix} 
 \leq 
 \begin{vmatrix}
 l_{lb,0}\\
 l_{lb,1}\\
 ...\\
 l_{lb,m}\\
 \end{vmatrix}
$$



Similarly, for upper boundary $l_{ub,j}$, Apollo calculates the inequality constraints as: 

$$
\begin{vmatrix} 
 1 & t_0 & t_0^2 & t_0^3 & t_0^4&t_0^5 \\
  1 & t_1 & t_1^2 & t_1^3 & t_1^4&t_1^5 \\
 ...&...&...&...&...&... \\
 1 & t_m & t_m^2 & t_m^3 & t_m^4&t_m^5 \\
 \end{vmatrix} \cdot \begin{vmatrix} a_i \\ b_i \\ c_i \\ d_i \\ e_i \\ f_i \end{vmatrix} 
 \leq
 -1 \cdot
 \begin{vmatrix}
 l_{ub,0}\\
 l_{ub,1}\\
 ...\\
 l_{ub,m}\\
 \end{vmatrix}
$$




--- 3.5  Speed Boundary constraint

Apollo establishes a speed limit boundary as well.

Sample **m** points on the st curve, and get speed limits defined as an upper boundary and a lower boundary for each point $j$, e.g., $v{ub,j}$ and $v{lb,j}$ . The constraints are defined as: 

$$
f'(t_j) \geq v_{lb,j}
$$

Namely

$$
\begin{vmatrix}  
0& 1 & t_0 & t_0^2 & t_0^3 & t_0^4 \\  
0 & 1 & t_1 & t_1^2 & t_1^3 & t_1^4 \\ 
...&...&...&...&...&... \\ 
0& 1 & t_m & t_m^2 & t_m^3 & t_m^4 \\ 
\end{vmatrix} 
\cdot 
\begin{vmatrix} 
a_i \\ b_i \\ c_i \\ d_i \\ e_i \\ f_i 
\end{vmatrix}  
\geq  
\begin{vmatrix} v_{lb,0}\\ v_{lb,1}\\ ...\\ v_{lb,m}\\ \end{vmatrix}
$$

And 

$$
f'(t_j) \leq v_{ub,j}
$$

Namely

$$
\begin{vmatrix} 
 0& 1 & t_0 & t_0^2 & t_0^3 & t_0^4 \\
 0 & 1 & t_1 & t_1^2 & t_1^3 & t_1^4 \\
 ...&...&...&...&...&... \\
 0 &1 & t_m & t_m^2 & t_m^3 & t_m^4 \\
 \end{vmatrix} \cdot \begin{vmatrix} a_i \\ b_i \\ c_i \\ d_i \\ e_i \\ f_i \end{vmatrix} 
 \leq
 \begin{vmatrix}
 v_{ub,0}\\
 v_{ub,1}\\
 ...\\
 v_{ub,m}\\
 \end{vmatrix}
$$
\end{document}  % 文档结束